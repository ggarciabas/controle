%% Exemplo de utilizacao do estilo de formatacao normas-utf-tex (http://normas-utf-tex.sourceforge.net)
%% Autores: Hugo Vieira Neto (hvieir@utfpr.edu.br)
%%          Diogo Rosa Kuiaski (diogo.kuiaski@gmail.com)
%% Colaboradores:
%%          Cézar M. Vargas Benitez <cesarvargasb@gmail.com>
%%          Marcos Talau <talau@users.sourceforge.net>


\documentclass[openright]{normas-utf-tex} %openright = o capitulo comeca sempre em paginas impares
%\documentclass[oneside]{normas-utf-tex} %oneside = para dissertacoes com numero de paginas menor que 100 (apenas frente da folha) 


\usepackage[alf,abnt-emphasize=bf,bibjustif,recuo=0cm, abnt-etal-cite=2, abnt-etal-list=99]{abntcite} %configuracao correta das referencias bibliograficas.

\usepackage[brazil]{babel} % pacote portugues brasileiro
%\usepackage[T1]{fontenc}
%\usepackage[latin1]{inputenc} % pacote para acentuacao direta
\usepackage[utf8]{inputenc} % pacote para acentuacao direta
\usepackage{abnt-alf}
\usepackage{amsmath,amsfonts,amssymb} % pacote matematico
\usepackage{graphicx} % pacote grafico
\usepackage{times} % fonte times
\usepackage{enumitem}

%Podem utilizar GEOMETRY{...} para realizar pequenos ajustes das margens. Onde, left=esquerda, right=direita, top=superior, bottom=inferior. P.ex.:
%\geometry{left=3.0cm,right=1.5cm,top=4cm,bottom=1cm} 

% ---------- Preambulo ----------
\instituicao{Universidade Tecnol\'ogica Federal do Paran\'a} % nome da instituicao
%\programa{Programa de P\'os-gradua\c{c}\~ao em Engenharia El\'etrica e Inform\'atica Industrial} % nome do programa
\programa{Curso Superior de Tecnologia em Análise e Desenvolvimento de Sistemas} % nome do programa
%\area{Inform\'atica Industrial} % [Engenharia Biom\'edica] ou [Inform\'atica Industrial] ou [Telem\'atica]
\area{Ci\^encia da Computa\c{c}\~ao} % [Engenharia Biom\'edica] ou [Inform\'atica Industrial] ou [Telem\'atica]
\documento{Trabalho}%{Disserta\c{c}\~ao} [Disserta\c{c}\~ao], [Tese] ou [TCC]
\nivel{Gradua\c{c}\~ao} % [Mestrado] ou [Doutorado]
\titulacao{Graduado} % [Mestre] , [Doutor],[Graduado]

\titulo{\MakeUppercase{Documento de requisitos}} % titulo do trabalho em portugues
%\title{\MakeUppercase{Title in English}} % titulo do trabalho em ingles

\autor{Giovanna Garcia Basilio, Marcos Agnes} % autor do trabalho
%\cita{SOBRENOME, Nome} % sobrenome (maiusculas), nome do autor do trabalho

\palavraschave{Palavra-chave 1, Palavra-chave 2, ...} % palavras-chave do trabalho
\keywords{Keyword 1, Keyword 2, ...} % palavras-chave do trabalho em ingles

\comentario{\UTFPRdocumentodata\ apresentado(a) ao \UTFPRprogramadata\ da \ABNTinstituicaodata\ como requisito parcial para obten\c{c}\~ao do grau de ``\UTFPRtitulacaodata\ em \UTFPRareadata'' -- \'Area de Concentra\c{c}\~ao: \UTFPRareadata.}



\orientador{Nome do Orientador} % nome do orientador do trabalho
%\orientador[Orientadora:]{Nome da Orientadora} % <- no caso de orientadora, usar esta sintaxe
%\coorientador{Nome do Co-orientador} % nome do co-orientador do trabalho, caso exista
%\coorientador[Co-orientadora:]{Nome da Co-orientadora} % <- no caso de co-orientadora, usar esta sintaxe
%\coorientador[Co-orientadores:]{Nome do Co-orientador} % no caso de 2 co-orientadores, usar esta sintaxe
%\coorientadorb{Nome do Co-orientador 2}	% este comando inclui o nome do 2o co-orientador

\local{Medianeira} % cidade
\data{\the\year} % ano automatico


%---------- Inicio do Documento ----------
\begin{document}

\capa % geracao automatica da capa
%\folhaderosto % geracao automatica da folha de rosto
%\termodeaprovacao % <- ainda a ser implementado corretamente

% dedicatória (opcional)
%\begin{dedicatoria}
%Texto da dedicat\'oria.
%\end{dedicatoria}

% agradecimentos (opcional)
%\begin{agradecimentos}
%Texto dos agradecimentos.
%\end{agradecimentos}

% epigrafe (opcional)
%\begin{epigrafe}
%Texto da ep\'igrafe.
%\end{epigrafe}

%resumo
%\begin{resumo}
%Texto do resumo (m\'aximo de 500 palavras).
%\end{resumo}

%abstract
%\begin{abstract}
%Abstract text (maximum of 500 words).
%\end{abstract}

% listas (opcionais, mas recomenda-se a partir de 5 elementos)
%\listadefiguras % geracao automatica da lista de figuras
\listadetabelas % geracao automatica da lista de tabelas
%\listadesiglas % geracao automatica da lista de siglas
%\listadesimbolos % geracao automatica da lista de simbolos

% sumario
\sumario % geracao automatica do sumario


%---------- Inicio do Texto ----------
% recomenda-se a escrita de cada capitulo em um arquivo texto separado (exemplo: intro.tex, fund.tex, exper.tex, concl.tex, etc.) e a posterior inclusao dos mesmos no mestre do documento utilizando o comando \input{}, da seguinte forma:
%\input{intro.tex}
%\input{fund.tex}
%\input{exper.tex}
%\input{concl.tex}


%---------- Primeiro Capitulo ----------
\chapter{Pref\'acio}

Deve apresentar umc controle de vers\~oes bem como uma justificativa acerca do desenvolvimento. Deve especificar os leitores a quem se destina esse documento.


\section{A quem destina-se este documento}
	


\section{Controle de vers\~oes}

\begin{table}[!htb]
	\centering	
	\label{tab:contVersao}
	\begin{tabular}{cccc}
		\hline 
		Data & Vers\~ao & Descriç\~ao & Autor \\
		\hline
		21/06/2013 & 1.0 & Versão console & Giovanna Garcia \\
		\hline
		21/06/2013 & 1.0 & Adicionado modelos e padrões & Giovanna Garcia, Marcos Agnes\\
		\hline
		21/06/2013 & 1.0 & Adicionado licença & Giovanna Garcia\\
		\hline 
	\end{tabular}
	\caption[Controle de vers\~oes]{Controle de vers\~oes.}
%	\fonte{Autoria pr\'opria.}
\end{table}


%---------- Segundo Capitulo ----------
\chapter{Introduç\~ao}

\section{Prop\'osito do Documento}
	Este documento cont\'em a especificaç\~ao de requisitos para o sistema CONSUN (CONtrole de Salas da UNiversidade Tecnol\'ogica Federal do Paran\'a), que gerenciará as reservas de sala.
	
\section{Escopo do Produto}
	O sistema tem como objetivo auxiliar no gerenciamento de reserva de salas da universidade, como: inserir, excluir, modificar e consultar.
	
\section{Definiç\~oes e Abreviaç\~oes}
	As definiç\~oes utilizadas neste documento ser\~ao abordadas posteriormente no gloss\'ario.	
	
	Abreviaç\~oes:	
		\begin{itemize}%[leftsep=2cm]
		    \item RG: respons\'avel geral;
		    \item RS: respons\'avel setor;
		    \item PF: professor;
			\item RF: requisito funcional;
			\item RNF: requisito n\~ao funcional.
		\end{itemize}	
		
		
\section{Vis\~ao Geral do documento}
	Este documento apresenta uma descriç\~ao geral do sistema, e logo em seguida descreve suas funcionalidades especificando as entradas e sa\'idas para todos os requisitos funcionais. Faz tamb\'em uma descriç\~ao sucinta dos requisitos n\~ao funcionais contidos neste sistema.
	




%---------- Terceiro Capitulo ------------
\chapter{Gloss\'ario}



%--------- Quarto Capitulo -------------
\chapter{Vis\~ao Geral}
%	O sistema CONSUN gerencia as reservas de salas da UTFPR (Universidade Tecnol\'ogica Federal do Paran\'a) , onde somente os usu\'arios logados ter\~ao %permiss\~ao para inserir, excluir, modificar e consultar as reservas. Neste caso somente o RG poder\'a inserir e excluir um usu\'ario e uma sala. O RS poder\'a %modificar e deletar todas as reservas, e modificar as salas de seu setor. PFs somente poder\~ao inserir, excluir e modificar a sua pr\'opria reserva, e modificar o %seu perfil de usu\'ario.
	O objetivo do sistema é gerenciar as reservas de salas, bem como os usu\'arios, tendo o controle dos n\'iveis de prioridade dentro da estrutura hier\'arquica do sistema, permitindo que determinado usu\'ario somente execute o que seu n\'ivel lhe permita. %Os dados serão armazenados em um banco de dados contido no servidor web.
	
	A reserva somente poder\'a ser feita por um usu\'ario previmente cadastrado no sistema. O Respons\'avel Geral \'e o único usuário que poderá inserir e deletar outros usuários. Cada usuário poderá somente modificar seu cadastro.
	O cadastro de salas \'e feito pelo Responsável Geral e os Responsáveis por cada setor terão a permissão para modificar as salas.
	
	Para que uma reserva seja feita a sala desejada deverá estar disponível na data determinada pelo requerente (usuário). O usuário poderá inserir e consultar qualquer reserva, excluir e modificar somente a sua reserva. Havendo algum problema com relação a reserva de uma determinada sala, o Responsável Setor poderá excluir e modificar as reservas de seu setor.
	
	
\section{Perspectiva do Produto}
	O sistema opera com uma m\'aquina servidor que gerencia os cadastro no banco de dados.
	
\section{Funç\~oes do Produto}
	Gerenciamento de reservas de salas: inserir, modificar, excluir e consultar.
	Gerenciamento de usu\'arios: inserir, modificar, excluir e consultar.
	Gerenciamento de salas: inserir, modificar, excluir e consultar.
	
\section{Restriç\~oes Gerais}
	O sistema n\~ao permitir\'a o acesso de pessoas n\~ao cadastradas no sistema.
	


%--------- Quinto Capitulo -----------
\chapter{Especificaç\~ao de requisitos}

	An\'alise de requisitos, faz parte da engenharia de software e tem por objetivo ligar alocação de software em n\'ivel de sistema com o projeto de software.\\
	Esta an\'alise da a possibilidade de especificar funções, desempenho do software, interface e restrições.\\
	Estes requisitos podem ser divididos em reconhecimento do problema, avaliação e s\'intese, modelagem, especificação e revisão.\\

\section{Requisitos funcionais e n\~ao funcionais}
	Da \ref{tab:rfcadusu} at\'e a \ref{tab:rfconsres} ser\~ao apresentados os requisitos funcionais e n\~ao funcionais do sistema  CONSUN.
		
\begin{table}[h!]
	\centering
	\begin{tabular}{|p{4cm}|p{4cm}|c|c|c|}
		\hline
		\multicolumn {1}{|p{4cm}|} {RF. 1 Cadastrar Usu\'ario} & \multicolumn{4}{|l|}{ Evidente()}\\
		\hline
		\multicolumn{5}{|p{14cm}|} {Descriç\~ao: Somente o RG poder\'a cadastrar os usu\'arios do sistema.}\\
		\hline
		\multicolumn{5}{|l|} {Requisitos não Funcionais}\\
		\hline
		Nome & Restrição & Categoria & Desej\'avel & Permanente \\
		\hline	
		1.1 Obter dados para cadastro & Nome do usu\'ario, e-mail e senha. & Interface & ( ) & (X) \\
		\hline	
		1.2 Cadastrar dados & Verificar a exist\^encia do usu\'ario no banco. & Segurança & ( ) & (X) \\
		\hline
		% Sa\'ida: Mensagem de confirmaç\~ao bem sucedido do cadastro caso tenha sido efetuado com sucesso, sen\~ao, mensagem de erro.\\
	\end{tabular}
	\caption{RF. 1 Cadastrar Usu\'ario}
	\label{tab:rfcadusu}
\end{table}	
		 	 
\begin{table}[h!]
	\centering
	\begin{tabular}{|p{4cm}|p{4cm}|c|c|c|}
		\hline
		\multicolumn {1}{|p{4cm}|} {RF. 2 Modificar Cadastro de Usu\'ario} & \multicolumn{4}{|l|}{ Evidente()}\\
		\hline
		\multicolumn{5}{|p{14cm}|} {Descriç\~ao: O PF pode fazer alteraç\~ao de seu perfil.}\\
		\hline
		\multicolumn{5}{|l|} {Requisitos não Funcionais}\\
		\hline
		Nome & Restrição & Categoria & Desej\'avel & Permanente \\
		\hline	
		2.1 Obter dados da modifcaç\~ao & Identificaç\~ao do usu\'ario, campo desejado e novo dado. & Interface & ( ) & (X)\\	
		\hline
		%  \indent \indent Processo: Atualizaç\~ao no banco de dados.\\
		%	 \indent \indent Sa\'ida: Mensagem de confirmaç\~ao bem sucedido da modificaç\~ao caso tenha sido efetuado com sucesso, sen\~ao, mensagem de erro.\\
	\end{tabular}
	\caption{RF. 2 Modificar Cadastro de Usu\'ario}
	\label{tab:rfmodfusu}
\end{table}			 	
			
			 
\begin{table}[h!]
	\centering
	\begin{tabular}{|p{4cm}|p{4cm}|c|c|c|}
		\hline
		\multicolumn {1}{|p{4cm}|} {RF. 3 Excluir Cadastro de Usu\'ario} & \multicolumn{4}{|l|}{ Evidente()}\\
		\hline
		\multicolumn{5}{|p{14cm}|} {Descriç\~ao: Somente o RG poder\'a excluir o cadastro do usu\'ario.}\\
		\hline
		\multicolumn{5}{|l|} {Requisitos não Funcionais}\\
		\hline
		Nome & Restrição & Categoria & Desej\'avel & Permanente \\
		\hline		
		3.1 Obter dados & Identificação do usu\'ario. & Interface & ( ) & (X)\\
		\hline
		3.2 Excluir sala & Verificar se o usu\'ario \'e cadastrado no sistema. & Segurança & ( ) & (X)\\
		\hline
		%  Sa\'ida: Mensagem de confirmaç\~ao bem sucedido da exclus\~ao do cadastro caso tenha sido efetuado com sucesso, sen\~ao, mensagem de erro.\\
	\end{tabular}
	\caption{RF. 3 Excluir Cadastro de Usu\'ario}
	\label{tab:rfexcusu}
\end{table}						
			 
\begin{table}[h!]
	\centering
	\begin{tabular}{|p{4cm}|p{4cm}|c|c|c|}
		\hline
		\multicolumn {1}{|p{4cm}|} {RF. 4 Consultar Usu\'ario} & \multicolumn{4}{|l|}{ Evidente()}\\
		\hline
		\multicolumn{5}{|p{14cm}|} {Descriç\~ao: Todos os usu\'arios poder\~ao efetuar a consulta.}\\
		\hline
		\multicolumn{5}{|l|} {Requisitos não Funcionais}\\
		\hline
		Nome & Restrição & Categoria & Desej\'avel & Permanente \\
		\hline		
		4.1 Obter dados & Identificaç\~ao do usu\'ario. & Interface & ( ) & (X)\\
		\hline
		4.2 Consultar usu\'ario & Verificar se o usu\'ario \'e cadastrado no sistema. & Segurança & ( ) & (X)\\
		\hline
		% Sa\'ida: Retorno dos dados do usu\'ario caso o mesmo esteja cadastrado no sistema, sen\~ao, mensagem de erro.\\ 			 			 	
	\end{tabular}
	\caption{RF. 4 Consultar Usu\'ario}
	\label{tab:rfconsusu}
\end{table}			
			 
\begin{table}[h!]
	\centering
	\begin{tabular}{|p{4cm}|p{4cm}|c|c|c|}
		\hline
		\multicolumn {1}{|p{4cm}|} {RF. 5 Cadastro de Salas} & \multicolumn{4}{|l|}{ Evidente()}\\
		\hline
		\multicolumn{5}{|p{14cm}|} {Descriç\~ao: Somente o RG poder\'a cadastrar as salas no sistema.}\\
		\hline
		\multicolumn{5}{|l|} {Requisitos não Funcionais}\\
		\hline
		Nome & Restrição & Categoria & Desej\'avel & Permanente \\
		\hline		
		5.1 Obter dados & Descrição da sala  e seus decoradores. & Interface & ( ) & (X)\\
		\hline
		% \indent \indent Processo: O cadastrp ser\'a incluido no banco de dados.\\
		%	 \indent \indent Sa\'ida: Mensagem de confirmaç\~ao bem sucedido do cadastro caso tenha sido efetuado com sucesso, sen\~ao, mensagem de erro.\\		
	\end{tabular}
	\caption{RF. 5 Cadastro de Salas}
	\label{tab:rfcadsala}
\end{table}					 			 
			 
\begin{table}[h!]
	\centering
	\begin{tabular}{|p{4cm}|p{4cm}|c|c|c|}
		\hline
		\multicolumn {1}{|p{4cm}|} {RF. 6 Modificar Cadastro de Sala} & \multicolumn{4}{|l|}{ Evidente()}\\
		\hline
		\multicolumn{5}{|p{14cm}|} {Descriç\~ao: Somente o RG e o RS	poder\~ao modificar as salas do sistema.}\\
		\hline
		\multicolumn{5}{|l|} {Requisitos não Funcionais}\\
		\hline
		Nome & Restrição & Categoria & Desej\'avel & Permanente \\
		\hline		
		6.1 Obter dados & Identificaç\~ao da sala, campo desejado e novo dado. & Interface & ( ) & (X)\\
		\hline
		%  \indent \indent Processo: Atualizaç\~ao no banco de dados.\\
		%	 \indent \indent Sa\'ida: Mensagem de confirmaç\~ao bem sucedido da modificaç\~ao caso tenha sido efetuado com sucesso, sen\~ao, mensagem de erro.\\
	\end{tabular}
	\caption{RF. 6 Modificar Cadastro de Sala}
	\label{tab:rfmodfsala}
\end{table}			 	 			 
				 			 			 			     		 			 						 
\begin{table}[h!]
	\centering
	\begin{tabular}{|p{4cm}|p{4cm}|c|c|c|}
		\hline
		\multicolumn {1}{|p{4cm}|} {RF. 7 Excluir Sala} & \multicolumn{4}{|l|}{ Evidente()}\\
		\hline
		\multicolumn{5}{|p{14cm}|} {Descriç\~ao: Somente o RG poder\'a excluir a sala.}\\
		\hline
		\multicolumn{5}{|l|} {Requisitos não Funcionais}\\
		\hline
		Nome & Restrição & Categoria & Desej\'avel & Permanente \\
		\hline		
		7.1 Obter dados & Identificaç\~ao da sala. & Interface & ( ) & (X)\\
		\hline
		7.2 Exluir sala & Verificar se a sala \'e cadastrada no sistema. & Segurança & ( ) & (X)\\
		\hline
		% 			 \indent \indent Sa\'ida: Mensagem de confirmaç\~ao bem sucedido da exclus\~ao do cadastro caso tenha sido efetuado com sucesso, sen\~ao, mensagem de erro.\\	
	\end{tabular}
	\caption{RF. 7 Excluir Sala}
	\label{tab:rfexcsala}
\end{table}					 

\begin{table}[h!]
	\centering
	\begin{tabular}{|p{4cm}|p{4cm}|c|c|c|}
		\hline
		\multicolumn {1}{|p{4cm}|} {RF. 8 Consultar Sala} & \multicolumn{4}{|l|}{ Evidente()}\\
		\hline
		\multicolumn{5}{|p{14cm}|} {Descriç\~ao: Todos os usu\'arios poderão efetuar a consulta.}\\
		\hline
		\multicolumn{5}{|l|} {Requisitos não Funcionais}\\
		\hline
		Nome & Restrição & Categoria & Desej\'avel & Permanente \\
		\hline		
		8.1 Obter dados & Identificaç\~ao da sala. & Interface & ( ) &(X)\\
		\hline
		8.2 Consultar sala & Verificar se a sala \'e cadastrada no sistema. & Segurança & ( ) & (X)\\
		\hline
		% 			 \indent \indent Sa\'ida: Retorno dos dados da sala caso a mesma esteja cadastrada no sistema, senão, mensagem de erro.\\	 
	\end{tabular}
	\caption{RF. 8 Consultar Sala}
	\label{tab:rfconssala}
\end{table}					 
			 
\begin{table}[h!]
	\centering
	\begin{tabular}{|p{4cm}|p{4cm}|c|c|c|}
		\hline
		\multicolumn {1}{|p{4cm}|} {RF. 9 Cadastrar Reserva de Sala} & \multicolumn{4}{|l|}{ Evidente()}\\
		\hline
		\multicolumn{5}{|p{14cm}|} {Descriç\~ao: Todos os usu\'arios poderão reservar uma sala.}\\
		\hline
		\multicolumn{5}{|l|} {Requisitos não Funcionais}\\
		\hline
		Nome & Restrição & Categoria & Desej\'avel & Permanente \\
		\hline		
		9.1 Obter dados & Identificaç\~ao da sala, identificaç\~ao do usu\'ario, data, hora in\'icio e hora fim. & Interface & ( ) & (X)\\
		\hline 
		9.2 Cadastrar reserva & Verificar se a sala \'e cadastrada no sistema e se a mesma j\'a possui uma reserva na data. & Segurança & ( ) & (X)\\
		\hline
		%			 \indent \indent Sa\'ida: Mensagem de confirmaç\~ao bem sucedido da reserva da sala caso tenha sido efetuado com sucesso, sen\~ao, mensagem de erro.\\
	\end{tabular}
	\caption{RF. 9 Cadastrar Reserva de Sala}
	\label{tab:rfcadres}
\end{table}					 
			 
\begin{table}[h!]
	\centering
	\begin{tabular}{|p{4cm}|p{4cm}|c|c|c|}
		\hline
		\multicolumn {1}{|p{4cm}|} {RF. 10 Excluir Reserva de Sala} & \multicolumn{4}{|l|}{ Evidente()}\\
		\hline
		\multicolumn{5}{|p{14cm}|} {Descriç\~ao: O RS poder\'a excluir qualquer reserva do seu setor. O PF poder\'a somente deletar sua reserva.}\\
		\hline
		\multicolumn{5}{|l|} {Requisitos não Funcionais}\\
		\hline
		Nome & Restrição & Categoria & Desej\'avel & Permanente \\
		\hline		
		10.1 Obter dados & Identificaç\~ao da reserva e identificaç\~ao do usu\'ario. & Interface & ( ) & (X)\\
		\hline
		10.2 Excluir reserva & Verificar se a reserva \'e cadastrada no sistema. & Segurança & ( ) & (X)\\
		\hline
		%			 \indent \indent Sa\'ida: Mensagem de confirmaç\~ao bem sucedido da reserva da sala caso tenha sido efetuado com sucesso, sen\~ao, mensagem de erro.\\
	\end{tabular}
	\caption{RF. 10 Excluir Reserva de Sala}
	\label{tab:rfexcres}
\end{table}					 
			 
\begin{table}[h!]
	\centering
	\begin{tabular}{|p{4cm}|p{4cm}|c|c|c|}
		\hline
		\multicolumn {1}{|p{4cm}|} {RF. 11 Modificar Reserva de Sala} & \multicolumn{4}{|l|}{ Evidente()}\\
		\hline
		\multicolumn{5}{|p{14cm}|} {Descriç\~ao: O RS poder\'a modificar qualquer reserva do seu setor. O PF poder\'a somente modificar sua reserva.}\\
		\hline
		\multicolumn{5}{|l|} {Requisitos não Funcionais}\\
		\hline
		Nome & Restrição & Categoria & Desej\'avel & Permanente \\
		\hline		
		11.1 Obter dados & Identificaç\~ao da reserva, do usu\'ario, campo desejado e novos dados. & Interface & ( ) & (X)\\
		\hline
		11.2 Modificar reserva & Verificar se a reserva \'e cadastrada no sistema. & Segurança & ( ) & (X)\\
		\hline
		% 			 \indent \indent Sa\'ida: Mensagem de confirmaç\~ao bem sucedido da reserva da sala caso tenha sido efetuado com sucesso, sen\~ao, mensagem de erro.\\
	\end{tabular}
	\caption{RF. 11 Modificar Reserva de Sala}
	\label{tab:rfmodfres}
\end{table}					 
			 
\begin{table}[h!]
	\centering
	\begin{tabular}{|p{4cm}|p{4cm}|c|c|c|}
		\hline
		\multicolumn {1}{|p{4cm}|} {RF. 12 Consultar Reserva de Sala} & \multicolumn{4}{|l|}{ Evidente()}\\
		\hline
		\multicolumn{5}{|p{14cm}|} {Descriç\~ao: Todos os usu\'arios cadastrados poder\~ao consultar as reservas de sala.}\\
		\hline
		\multicolumn{5}{|l|} {Requisitos não Funcionais}\\
		\hline
		Nome & Restrição & Categoria & Desej\'avel & Permanente \\
		\hline		
		12.1 Obter dados & Identificaç\~ao da reserva de sala. & Interface & ( ) & (X)\\
		\hline
		12.2 Consultar reserva & Verificar se a reserva \'e cadastrada no sistema. & Segurança & ( ) & (X)\\
		\hline
		%			 \indent \indent Sa\'ida: Retorno dos dados da reserva caso a mesma esteja cadastrada no sistema, sen\~ao, mensagem de erro.\\
	\end{tabular}
	\caption{RF. 12 Consultar Reserva de Sala}
	\label{tab:rfconsres}
\end{table}					 									 			 					 			 			 					 			 			 	
%\chapter{Arquitetura e modelos do sistema}

Dever\~ao ser apresentados modelos gr\'aficos dos sitema. Especificaç\~oes de linguagem ou banco de dados podem ser apresentadas (opcionalmente).

\subsection{Diagrama de caso de uso}


\subsection{Descriç\~ao dos casos de uso}

\subsection{Diagrama de sequ\^encia}

\subsection{Diagrama de classes}




%-------- Sexto Capitulo -----------
\chapter{evoluç\~ao do sistema}

Apresentar poss\'iveis mudanças/evoluç\~ao em funç\~ao de dificuldades t\'ecnicas/financeiras ou mesmo temporais.


%---------- Referencias ----------
%\bibliography{reflatex} % geracao automatica das referencias a partir do arquivo reflatex.bib


%-------- Apendice ------------
\apendice
\chapter{Nome ap\^endice}


% ---------- Anexos (opcionais) ----------
\anexo
\chapter{Nome do Anexo}

Use o comando {\ttfamily \textbackslash anexo} e depois comandos {\ttfamily \textbackslash chapter\{\}}
para gerar t\'itulos de anexos.


% --------- Lista de siglas --------
%\textbf{* Observa\c{c}\~oes:} a lista de siglas nao realiza a ordenacao das siglas em ordem alfabetica
% Em breve isso sera implementado, enquanto isso:
%\textbf{Sugest\~ao:} crie outro arquivo .tex para siglas e utilize o comando \sigla{sigla}{descri\c{c}\~ao}.
%Para incluir este arquivo no final do arquivo, utilize o comando \input{arquivo.tex}.
%Assim, Todas as siglas serao geradas na ultima pagina. Entao, devera excluir a ultima pagina da versao final do arquivo
% PDF do seu documento.


%-------- Citacoes ---------
% - Utilize o comando \citeonline{...} para citacoes com o seguinte formato: Autor et al. (2011).
% Este tipo de formato eh utilizado no comeco do paragrafo. P.ex.: \citeonline{autor2011}

% - Utilize o comando \cite{...} para citacoeses no meio ou final do paragrafo. P.ex.: \cite{autor2011}



%-------- Titulos com nomes cientificos (titulo, capitulos e secoes) ----------
% Regra para escrita de nomes cientificos:
% Os nomes devem ser escritos em italico, 
%a primeira letra do primeiro nome deve ser em maiusculo e o restante em minusculo (inclusive a primeira letra do segundo nome).
% VEJA os exemplos abaixo.
% 
% 1) voce nao quer que a secao fique com uppercase (caixa alta) automaticamente:
%\section[nouppercase]{\MakeUppercase{Estudo dos efeitos da radiacao ultravioleta C e TFD em celulas de} {\textit{Saccharomyces boulardii}}
%
% 2) por padrao os cases (maiusculas/minuscula) sao ajustados automaticamente, voce nao precisa usar makeuppercase e afins.
% \section{Introducao} % a introducao sera posta no texto como INTRODUCAO, automaticamente, como a norma indica.


\end{document}