\chapter{Introduç\~ao}
	\'E proposto o desenvolvimento do sistema CONSUN (Controle de salas da Universidade Tecnológica Federal do Paran\'a), que irá informatizar as funções de controle de cadastros de salas e usuários, e reserva de salas.
		
	Atualmente, a reserva de sala é feita através de planilhas ou de forma verbal. Quando um professor deseja reservar uma sala, ele precisa se deslocar até a sala de apoio ao aluno onde a mesma é realizada. 
	
	Assim, a partir de qualquer dispositivo conectado a rede, ele poderá acessar o sistema e visualizar todas as salas e suas características. Sendo assim, ele poderá reservar a sala que melhor atenda as suas necessidades com maior agilidade e praticidade.	
	

\section{Prop\'osito do Documento}
	Este documento cont\'em a análise do sistema CONSUN (CONtrole de Salas da UNiversidade Tecnol\'ogica Federal do Paran\'a), que gerenciará as reservas de sala. Descrevendo visão geral, requisitos funcionais e não funcionais e diagramas de caso de uso, sequência, atividades e classes.
	
\section{Escopo do Produto}
	O sistema tem como objetivo auxiliar no gerenciamento de reserva de salas da universidade, como: inserir, excluir, modificar e consultar.
	
\section{Definiç\~oes e Abreviaç\~oes}
	As definiç\~oes utilizadas neste documento ser\~ao abordadas posteriormente no gloss\'ario.	
	
	Abreviaç\~oes:	
		\begin{itemize}%[leftsep=2cm]
		    \item RG: respons\'avel geral;
		    \item RS: respons\'avel setor;
		    \item PF: professor;
			\item RF: requisito funcional;
			\item RNF: requisito n\~ao funcional.
		\end{itemize}			
		
%\section{Vis\~ao Geral do documento}
%	Este documento apresenta uma descriç\~ao geral do sistema, e logo em seguida descreve suas funcionalidades especificando as %entradas e sa\'idas para todos os requisitos funcionais. Faz tamb\'em uma descriç\~ao sucinta dos requisitos n\~ao funcionais %contidos neste sistema.
	

